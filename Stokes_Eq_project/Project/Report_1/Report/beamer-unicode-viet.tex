\documentclass{beamer}
%\textwidth=9.5in 
%\textheight=6in 
%\topmargin -.7in
\mode<presentation>{\usetheme{default}\setbeamercovered{invisible}\usecolortheme{crane}}
\usepackage[utf8]{inputenc} 
\usepackage[vietnam]{babel}
\usepackage{xcolor}
\usepackage{amstext,latexsym,amsbsy,amssymb,amsmath,amsthm}
\pagenumbering{arabic}

% Định nghĩa khoảng cách, các bạn có thể xóa bỏ phần này

\newcommand{\hh}{\hspace*{.48in}}
\newcommand{\hz}{\hspace*{.24in}}
\newcommand{\hhh}{\hspace*{.72in}}
\newcommand{\nh}{\hspace*{-.12in}}
\newcommand{\nhh}{\hspace*{-.24in}}

% Định nghĩa các tiêu đề

\renewcommand{\chaptername}{Chương}
\renewcommand{\appendixname}{PHỤ LỤC}
\newcommand{\hty}{\rightharpoonup}
\newcommand{\rong}{\emptyset}
\newtheorem{tm}{ }
\newtheorem{dn}{\large Định nghĩa}[section]
\newtheorem{dly}{\large Định lý}[section]
\newtheorem{chy}{\large Chú ý}[section]
\newtheorem{bd}{\large Bổ đề}[section]
\newtheorem{md}{\large Mệnh đề}[section]
\newtheorem{hq}{\large Hệ quả}[section]
\newtheorem{ly}{\large Lưu ý}[section]
\newtheorem{thm}{Theorem}[section]
\newtheorem{lem}[thm]{Lemma}
\newtheorem{eg}[thm]{Example}
\newtheorem{prop}[thm]{Proposition}
\newtheorem{cor}[thm]{Corollary}
\newtheorem{defn}[thm]{Definition}
\newtheorem{rem}[thm]{Remark}
\newtheorem{ntn}[thm]{Notation}
\newenvironment{prf}{{\noindent \textbf{Proof:}\ }}{\hfill $\Box$\\ \smallskip}
\numberwithin{equation}{section}


% Định nghĩa các công thức toán, các bạn có thể xóa bỏ phần này

\def\inty{\infty}

\def\beqn{\begin{eqnarray}}\def\eeqn{\end{eqnarray}}
\def\bqq{\[}\def\eqq{\]}
\def\lef{\lefteqn}
\def\z{\textstyle }
\newcommand{\da}[3]{#1\zoz{#2}\zuz{#3}}
\newcommand{\ssm}[2]{\displaystyle\sum_{\mbox{\scriptsize$\begin{array}{l}
\mbox{\scriptsize$#1$}\\\mbox{\scriptsize$#2$}\end{array}$}}}
\newcommand{\lf}[2]{\mbox{\large$\frac{#1}{#2}$}}
\newcommand{\Lf}[2]{{\displaystyle\frac{\z #1}{\z #2}}}
\newcommand{\zu}[1]{_{{}_{\z #1}}}
\newcommand{\zo}[1]{^{\z #1}}
\newcommand{\zuz}[1]{_{\z{}_{#1}}}
\newcommand{\zoz}[1]{^{#1}}
\def\NN{\mbox{$I\hspace{-.06in}N$}}
\def\ZZ{\mbox{$Z\hspace{-.08in}Z\hspace{.05in}$}}
\def\QQ{\mbox{$Q\hspace{-.11in}\protect\raisebox{.5ex}{\tiny$/$}
\hspace{.06in}$}}
\def\RR{\mbox{$I\hspace{-.06in}R$}}
\def\CC{\mbox{$C\hspace{-.11in}\protect\raisebox{.5ex}{\tiny$/$}
\hspace{.06in}$}}
\def\HH{\mbox{$I\hspace{-.06in}H$}}
\def\bu{\mbox{$\bullet$}}
\def\jnt{\displaystyle\int}
\def\xo{\vspace{.05in}\\\hspace*{5in}\mbox{\huge$\Box$} }
\newcommand{\ii}[1]{\displaystyle\int_{{}_{\z#1}}}
\newcommand{\ik}[2]{\displaystyle\int_{{}_{\z #1}}^{\z #2}}
\def\io{\displaystyle\int_{{}_{\z \Omega}}}
\def\irn{\ii{\RR^{n}}}
\newcommand{\iii}[1]{\displaystyle\int\hspace{-.16in}-\hspace{-.06in}\zu{{}\zu{\z#1}}}
\def\imr{\iii{B(x,3r)}}
\newcommand{\iR}[1]{\displaystyle\int_{{}_{\z \RR^{#1}}}}
\newcommand{\ijz}[2]{\displaystyle\int_{{}_{ #1}}^{ #2}}
\newcommand{\khog}[4]{#1\zuz{#2}\zoz{#3}(#4)}
\newcommand{\vh}[2]{\langle #1,#2\rangle}
\def\sij{\displaystyle\sum_{i,j=1}^{n}} \def\sj{\displaystyle\sum_{j=1}^{n}}
\newcommand{\su}[2]{\displaystyle\sum\zu{#1}\zo{#2}}
\newcommand{\suz}[2]{\displaystyle\sum\zuz{#1}\zoz{#2}}
\newcommand{\suu}[4]{\displaystyle\sum_{#1}^{#2}\displaystyle\sum_{#3}^{#4}}
\newcommand{\sss}[3]{\displaystyle\sum_{\begin{array}{l}\mbox{\W$#1$}\\
\mbox{\W$#3$}\end{array}}^{#2}}
\newcommand{\ssso}[3]{{\displaystyle\sum\zu{#1}\zo{#2}}\hspace{-.3in}\zu{{}\zu{
{}\zu{{}\zu{{}\zu{{}\zu{\zu{{}_{#3}}}}}}}}\hspace{-.12in}}
\newcommand{\sssss}[5]{{\displaystyle\sum_{#1}^{#2}\displaystyle\sum_{#3}^{#4}}
\hspace{-.5in}_{{}\zu{{}\zu{{}\zu{{}\zu{{}\zu{{}\zu{{}_{#5}}}}}}}}}
\newcommand{\ca}[1]{{\cal#1}}


%Định nghĩa màu cho chữ viết, các bạn không nên  xóa bỏ phần này

\newcommand{\doo}[1]{\textcolor{red}{#1}}
\newcommand{\xanh}[1]{\textcolor{green}{#1}}
\newcommand{\duong}[1]{\textcolor{blue}{#1}}
\newcommand{\bich}[1]{\textcolor{cyan}{#1}}
\newcommand{\hong}[1]{\textcolor{magenta}{#1}}
\newcommand{\vang}[1]{\textcolor{yellow}{#1}}

%Định nghĩa màu cho khung và chữ viết, các bạn không nên  xóa bỏ phần này
\setbeamercolor{vangxanh}{fg=blue,bg=yellow!20!white}
\setbeamercolor{vangden}{fg=black,bg=yellow!20!white}
\setbeamercolor{vangdo}{fg=red,bg=yellow!20!white}
\setbeamercolor{camden}{fg=black,bg=orange!20!white}
\setbeamercolor{camxanh}{fg=blue,bg=orange!20!white}
\setbeamercolor{bichdo}{fg=red,bg=cyan!15!white}
\setbeamercolor{bichden}{fg=black,bg=cyan!15!white}
\setbeamercolor{hongxanh}{fg=blue,bg=magenta!15!white}
\setbeamercolor{hongden}{fg=black,bg=magenta!15!white}
\setbeamercolor{lado}{fg=red,bg=lime!40}
\setbeamercolor{laden}{fg=black,bg=lime!40}
\setbeamercolor{laxanh}{fg=blue,bg=lime!40}

% Định nghĩa một số ký hiệu toán, các bạn có thể xóa bỏ phần này

\newcommand{\id}{{\rm id}}
\newcommand{\ti}{\tilde}
\newcommand{\la}{\langle}
\newcommand{\ra}{\rangle}
\newcommand{\cb}{{\rm cb}}
\newcommand{\CB}{{\rm CB}}
\newcommand{\abs}[1]{\left\vert#1\right\vert}
\newcommand{\im}{{\rm Im\ \!}}
\newcommand{\sph}{\mathfrak{S}_1}
\newcommand{\CS}{\mathcal{S}}
\newcommand{\CK}{\mathcal{K}}
\newcommand{\CL}{\mathcal{L}}
\newcommand{\OX}{\mathbf{X}} 
\newcommand{\OY}{\mathbf{Y}}
\newcommand{\OZ}{\mathbf{Z}}
\newcommand{\BK}{\mathbf{K}}
\newcommand{\QS}{\mathcal{Q}}
\newcommand{\PS}{\mathcal{P}}
\newcommand{\QM}{{\rm qM}}
\newcommand{\PM}{{\rm pMor}}
\newcommand{\M}{{\rm Mor}}
\newcommand{\st}{{\empty^\ast}}
\newcommand{\reg}{{\rm reg}}
\newcommand{\bl}{\left}
\newcommand{\br}{\right}
\newcommand{\es}{{\rm es}}
\newcommand{\bal}{\mathfrak{B}_1}
\newcommand{\BN}{\mathbb{N}}



\title{\doo{\bf TỰA BÀI THUYẾT TRÌNH}}

\author{\xanh{Tên người thuyết trình}}
\institute{\duong{
Cơ quan đang công tác\\
Đại học hoặc viện đang công tác}}

\date{\bich{Báo cáo nghiệm thu đề tài cấp , hoặc Hội nghị Khoa Học A}}

\begin{document}

\begin{frame}
  \titlepage
\end{frame}

\begin{frame}
% frame này là danh sách của các tựa của  các \section có trong bài thuyết trình
 \frametitle{\duong{\LARGE \bf  Nội dung}}
 \tableofcontents
\end{frame}
\section{Cách dùng hướng dẫn này}
\begin{frame}
\underline{ \duong{\Large Cách dùng hướng dẫn này}} % tựa của slide tương ứng với frame này 
\begin{tm}\hong{ Chương trình Beamer dựa trên nền tảng Latex, cho phép chúng ta lập một pdf- file có hiệu ứng animation và font chữ Unicode-Việt.  Các bạn phải học cách sử dụng Latex trước.}\end{tm}

\begin{tm}\duong{So sánh bản tex-file "beamer-unicode-viet.tex" với bản pdf-file "beamer-unicode-viet.pdf", các bạn sẽ biết cách tạo bài trình bày bằng chương trình Beamer.}\end{tm}
\end{frame}
\section{Tạo slides}
\begin{frame}
\underline{ \duong{\Large Tạo slides}} % tựa của slide tương ứng với frame này 

\medskip    %\medskip dùng để chen các hàng để trống

\hong{ một slide được tạo bởi $\setminus$begin$\{$frame$\}\{$nội dung slide$\}\setminus$end$\{$frame$\}$
Mỗi frame sẽ tương ứng với một slide, vì thế ta nên liệu nội dung mỗi frame vừa với một slide. }
 % \
\medskip

\bich{ \Large {\bf Dùng  $\setminus$medskip  để tách rời các đoạn văn}} \\

\end{frame}

\section{Tô màu}
\begin{frame}
\underline{ \duong{\Large Tô màu}} % tựa của slide tương ứng với frame này 

\medskip

\duong{$\setminus$duong$\{$đoạn văn$\}$ để biến các chữ trong đoạn văn này có màu xanh dương}

\medskip

\doo{$\setminus$doo$\{$đoạn văn$\}$ để biến các chữ trong đoạn văn này có màu đỏ}

\medskip

\vang{ $\setminus$vang$\{$đoạn văn$\}$ để biến các chữ trong đoạn văn này có màu vàng}

\medskip

\xanh{ $\setminus$xanh$\{$đoạn văn$\}$ để biến các chữ trong đoạn văn này có màu xanh lá cây}

\medskip

\bich{ $\setminus$bich$\{$đoạn văn$\}$ để biến các chữ trong đoạn văn này có màu xanh ngọc bích}

\medskip

\hong{ $\setminus$hong$\{$đoạn văn$\}$ để biến các chữ trong đoạn văn này có màu hồng}
\medskip

\end{frame}


\begin{frame}

	\begin{beamercolorbox}[sep=.1em,wd=4.8in]{vangxanh}
  		Khi muốn tô "nền vàng,chữ xanh" một khung  có chiều ngang 4.8in, ta dùng $\setminus$begin$\{$beamercolorbox$\}\{$đoạn văn$\}\setminus$end$\{$beamercolorbox$\}$[sep=.1em,wd=4.8in]$\{$vangxanh$\}\{$đoạn văn$\}\setminus$end$\{$beamercolorbox$\}$\\
  		Trong đó các bạn có thể thay đổi bề ngang của khung tô màu bằng cách thay đổi tham số $wd=$. Trong một khung, chúng ta có thể chứa nhiều dòng kể cả các công thức. thí dụ \\
 \begin{equation}
\int_{\Omega}\nabla v\nabla\varphi dx=\int_{\Omega}f(x,v)\varphi dx,\forall\varphi\in W^{1,2}_0(\Omega).
\end{equation}
\end{beamercolorbox}

\medskip

\begin{beamercolorbox}[sep=.1em,wd=4.3in]{vangden}
  		Khi muốn tô "nền vàng,chữ đen" một khung  có chiều ngang 4.3in, ta dùng $\setminus$begin$\{$beamercolorbox$\}$[sep=.1em,wd=4.3in]$\{$vangden$\}$\\
  		\hh đoạn văn\\
  		$\setminus$end$\{$beamercolorbox$\}$	
\end{beamercolorbox}
\end{frame}


\begin{frame}

\begin{beamercolorbox}[sep=.1em,wd=4.3in]{vangdo}
"wd=4.3in" và  $\{$vangdo$\}$ cho ta  "nền vàng, chữ đỏ" .
\end{beamercolorbox}

\medskip  		

\begin{beamercolorbox}[sep=.1em,wd=4in]{camden}
  	"wd=4in" và  $\{$camden$\}$ cho ta  "nền cam, chữ đen" .
\end{beamercolorbox}
\medskip  		

\begin{beamercolorbox}[sep=.1em,wd=4.8in]{bichdo}
  	"wd=4.8in" và  $\{$bichdo$\}$ cho ta  "nền xanh ngọc bich, chữ đỏ" .
\end{beamercolorbox}
\medskip  		

\begin{beamercolorbox}[sep=.1em,wd=4.8in]{bichden}
  	"wd=4.8in" và  $\{$bichden$\}$ cho ta  "nền xanh ngọc bich, chữ đen" .
\end{beamercolorbox}
\medskip  		

\begin{beamercolorbox}[sep=.1em,wd=3.5in]{hongxanh}
  	"wd=3.5in" và  $\{$hongxanh$\}$ cho ta  "nền hồng, chữ xanh" .
\end{beamercolorbox}

\medskip  		

\begin{beamercolorbox}[sep=.1em,wd=3.5in]{hongden}
  	"wd=3.5in" và  $\{$hongden$\}$ cho ta  "nền hồng, chữ đen" .
\end{beamercolorbox}

\medskip  		

\begin{beamercolorbox}[sep=.1em,wd=4.5in]{lado}
  	"wd=4.5in" và  $\{$lado$\}$ cho ta  "nền xanh lá cây, chữ đỏ" .
\end{beamercolorbox}

\medskip  		

\begin{beamercolorbox}[sep=.1em,wd=4.5in]{laden}
  	"wd=4.5in" và  $\{$laden$\}$ cho ta  "nền xanh lá cây, chữ đen" .
\end{beamercolorbox}

\medskip  		

\begin{beamercolorbox}[sep=.1em,wd=4.5in]{laxanh}
  	"wd=4.5in" và  $\{$laxanh$\}$ cho ta  "nền xanh lá cây, chữ xanh" .
\end{beamercolorbox}

\end{frame}
\begin{frame}
	\begin{beamerboxesrounded}[upper=block head,lower=block body,shadow=true]{}
  		Khi muốn tô nền có chiều dài chuẩn  một khung đoạn văn, ta dùng \\
  	$\setminus$begin$\{$beamerboxesrounded$\}$\\$[upper=block head,lower=block body,shadow=true]\{\}$ \\	
  		 đoạn văn \\
  	$\setminus$end$\{$beamerboxesrounded$\}$
  	\end{beamerboxesrounded}

\medskip

	\begin{beamerboxesrounded}[upper=block head,lower=block body,shadow=true]{}
  		$\hh
\psi (x) = \left\{ \begin{array}{l}
 1~~\hhh|x| < 1 \vspace{.04in}\\ 
\in [0,1]\hh 1\le |x| \le \frac{3}{2}\vspace{.04in}\\
 0~~\hhh|x| > \frac{3}{2}. 
 \end{array} \right.$.
	\end{beamerboxesrounded}
\end{frame}





\section{Animation}
\begin{frame}
\underline{ \duong{\Large Animation}} % tựa của slide tương ứng với frame này 


\medskip
\pause % khi đặt \pause trước một đoạn nào, ta cần bấm enter  để đoạn đó hiện ra

\begin{beamercolorbox}[sep=.1em,wd=4.5in]{hongden}
 Khi đặt $\setminus$pause trước một đoạn nào, ta cần bấm enter  để đoạn đó hiện ra
 \end{beamercolorbox}
\medskip
\pause
 \duong{Nếu không có $\setminus$pause đoạn văn sẽ tự động hiện ra, không cần chờ ta bấm enter}\\
\medskip
\pause
\begin{beamercolorbox}[sep=.1em,wd=4.5in]{vangxanh}
 Ta có thể dùng $\setminus$pause  trong các bảng \vspace{.2in}\\
$\begin{tabular}{l|cccc}
Class & A & B & C & D \\\hline
X & 1 & 2 & 3 & 4 \pause\\
Y & 3 & 4 & 5 & 6 \pause\\
Z & 5 & 6 & 7 & 8
\end{tabular}$
\end{beamercolorbox}

\medskip
\begin{beamercolorbox}[sep=.1em,wd=4.5in]{hongden}
 Ta có thể dùng $\setminus$pause cho các phần  trong một câu \vspace{.2in}\\
  Để chứng minh điều này  \pause  ta dùng phản chứng
\end{beamercolorbox}
\end{frame}

\section{Hoàn tất file trình bày}
\begin{frame}
\underline{ \duong{\Large Hoàn tất file trình bày}} % tựa của slide tương ứng với frame này 

\medskip

\begin{beamercolorbox}[sep=.1em,wd=4.5in]{laden}
File trình bày hoàn tất là pdf-file đặc biệt, có những đoạn chúng ta phải bấm enter nó mới hiện ra. Đây là hiệu ứng animation giống như powerpoint-files. Sau khi chuẩn bị xong tex-file, để có pdf-file, ta làm các bước sau	\end{beamercolorbox}

\medskip
\begin{beamercolorbox}[sep=.1em,wd=4.5in]{vangxanh}
Vào {\bf tools} (trong texmaker) $->$  "{\bf LaTex}" , hoặc bấm {\bf F2}.
\end{beamercolorbox}
\medskip
\begin{beamercolorbox}[sep=.1em,wd=4.5in]{hongden}
Vào {\bf tools} (trong texmaker) $->$ "{\bf Dvi $->$ PS}" , hoặc bấm {\bf F4}.
\end{beamercolorbox} 
 \medskip
\begin{beamercolorbox}[sep=.1em,wd=4.5in]{bichdo} 
 Đóng tất cả các pdf-files đang mở trên computer của bạn. Vào {\bf tools} (trong texmaker) $->$ "{\bf PS $->$ PDF}" , hoặc bấm {\bf F8}, ta sẽ có pdf-file đặc biệt cần tìm.\end{beamercolorbox}

  \medskip
 \begin{beamercolorbox}[sep=.1em,wd=4.5in]{laxanh} 
  Vào {\bf tools} (trong texmaker) $->$  "{\bf View PDF}" , hoặc bấm {\bf F7} để xem pdf-file. Ta có thể dùng  pdf-file này bằng chương trình Acrobat bình thường để trình bày.\end{beamercolorbox}
  \end{frame}
\end{document}


